	\newpage
\section{Projektowanie}		%3
%Opis przygotowania narzędzi (git, visual studio). Wybór i opis bibliotek, klas. Szkic layoutów. Pseudo kody. Opisy wykorzystanych algorytmów (np. algorytm sortowania). Dokładniejsze określenie założeń i działania aplikacji, (np.: ten przycisk otworzy takie okno a w tym oknie wpisujemy takie dane).

\subsection{Opis przygotowania narzędzi}
% tu trzeba bedzie rypnac wszystko o android studio, nwm czy trzeba gita opisywac

\subsection{Założenie programu}
% Uznaje ze tutaj ma byc opisane co dokladnie robi interfejs?
W tym rodziale przedstawiona zostanie ogólna zasada działania programu.

Głównym celem programu jest odtwarzanie muzyki.

\subsection{Przedstawienie menu}
% moze to trzeba dac do zalozen programu

Program składa się z trzech okien.

\begin{itemize}
	\item Okno wyboru autora - AuthorsView()
	\item Okno wyboru albumu - AuthorView()
	\item Okno wyboru utworów - SongView()
\end{itemize}

Aplikacja włączając się wyświetla menu wyboru autora. Menu przedstawione jest w postaci kafelkowej.

Po wybraniu autora włączane jest menu wyboru albumu.

Po wybraniu albumu otwirane jest menu wyboru piosenek należących do tego utworu.

\subsection{Odczyt i przetwarzanie plików}

% baze danych tu sie wyrypnie 

Aplikacja będzie posiadać bazę danych. Baza danych będzie się składać z trzech tabelek.

\begin{itemize}
	\item Tabelka Autorów
	\item Tabelka Albumów
	\item Tabelka Utworów
\end{itemize}

W bazie danych będą istniały powiązania. W tabelce autorów będą trzy kolumny:
\begin{itemize}
	\item id\_aut
	\item name
	\item id\_album
\end{itemize}
Id\_aut zawiera id autora, name to nazwa autora, id\_album zawiera id albumu. Id będą ptrzebne do utworzenia powiązań.

W tabelce Albumów będą cztery kolumny:
\begin{itemize}
	\item id\_album
	\item name
	\item id\_song,
	\item id\_aut
\end{itemize}
Jak powyżej, id\_album to id albumu, name to nazwa, id\_sog to id piosenki, id\_aut to id autora.

W tabelce utworów będzą cztery kolumny
\begin{itemize}
	\item id\_song
	\item name
	\item id\_album
	\item id\_aut
\end{itemize}

Jak powyżej, id\_song to id piosenki, name to nazwa, id\_album to id albumu, id\_aut to id autora.

Id są potrzene do utworzenia powiązań między tabelkami.

Tabelka autorów będzie odwoływała się do tabelki albumów za pomocą id albumu. Tabelka albumów będzie odwoływała się do tabelki piosenek przy pomocy id song. Tabelka piosenek będzie posiadała odwołania do tabelek autorów i albumów za pomocą id aut i id album.

\subsection{Strukutura bazy danych}

\subsubsection{Ogólny opis}

Baza danych jest złożona z trzech tabel:

\begin{itemize}
	\item Autorzy - tabela ta ma zawierać wszystkie informacje o autorach z biblioteki użytkownika. Założeniem jest, że każdy autor ma unikalną nazwę, ponieważ nie ma żadnego dobrego sposobu unikalnej identyfikacji autorów z samych lokalnych plików.

	\item Albumy - tabela ta, oprócz katalogowania albumów, głównie pełni rolę \enquote{pośrednika} między piosenkami a autorami. Ważną informacją jaką zawiera każdy rekord, jest odnośnik do okładki danego albumu. Opisane jest to w sekcji nr.~\ref{sec:dbrelations}. Warto wspomnieć, że albumy każdego autora muszą mieć unikalne nazwy - problem identyfikacji jest podobny jak przy autorach - lecz nazwy albumów różnych autorów mogą się powtarzać. 
	
	\item Piosenki - tabela ta zawiera informacje o wszystkich piosenkach w bibliotece, pozyskane z tagów plików.
\end{itemize}

Detale dotyczące każdej z tabel można przeczytać w sekcji nr. \todo

\subsubsection{Relacje w bazie} \label{sec:dbrelations}

\begin{figure}[H]
	\centering
	\includegraphics[width=1\textwidth]{images/ch3-db-relacje.drawio.png}
	\caption{\centering{Model relacji bazy}}
	\label{fig:dbrelations}
\end{figure}

Na rysunku nr.~\ref{fig:dbrelations} ukazany jest uproszczony model bazy danych, biorący tylko pod uwagę komponenty potrzebne do określenia relacji. Autorzy są w relacji $M$ do $N$ z albumami. Każdy autor może mieć wiele albumów, a każdy album wiele autorów. Piosenki z albumami są w relacji $1$ do $N$. Każda piosenka może należeć wyłącznie do jednego albumu, ale każdy album może mieć wiele piosenek.

Warto zauważyć, że piosenki niebezpośrednio łączą się z autorami. Jeżeli piosenka chce uzyskać swojego autora, musi zrobić to poprzez album.
