	\newpage
\section{Projektowanie}		%3
%Opis przygotowania narzędzi (git, visual studio). Wybór i opis bibliotek, klas. Szkic layoutów. Pseudo kody. Opisy wykorzystanych algorytmów (np. algorytm sortowania). Dokładniejsze określenie założeń i działania aplikacji, (np.: ten przycisk otworzy takie okno a w tym oknie wpisujemy takie dane).

\subsection{Opis przygotowania narzędzi}
% tu trzeba bedzie rypnac wszystko o android studio, nwm czy trzeba gita opisywac

\subsection{Założenie programu}
% Uznaje ze tutaj ma byc opisane co dokladnie robi interfejs?
W tym rodziale przedstawiona zostanie ogólna zasada działania programu.

Głównym celem programu jest odtwarzanie muzyki.

\subsection{Przedstawienie menu}
% moze to trzeba dac do zalozen programu

Program składa się z trzech okien.

\begin{itemize}
	\item Okno wyboru autora - AuthorsView()
	\item Okno wyboru albumu - AuthorView()
	\item Okno wyboru utworów - SongView()
\end{itemize}

Aplikacja włączając się wyświetla menu wyboru autora. Menu przedstawione jest w postaci kafelkowej.

Po wybraniu autora włączane jest menu wyboru albumu.

Po wybraniu albumu otwirane jest menu wyboru piosenek należących do tego utworu.

\subsection{Odczyt i przetwarzanie plików}

\subsection{Strukutura bazy danych}

\subsubsection{Ogólny opis}

Baza danych jest złożona z trzech tabel:

\begin{itemize}
	\item Autorzy - tabela ta ma zawierać wszystkie informacje o autorach z biblioteki użytkownika. Założeniem jest, że każdy autor ma unikalną nazwę, ponieważ nie ma żadnego dobrego sposobu unikalnej identyfikacji autorów z samych lokalnych plików.

	\item Albumy - tabela ta, oprócz katalogowania albumów, głównie pełni rolę \enquote{pośrednika} między piosenkami a autorami. Ważną informacją jaką zawiera każdy rekord, jest odnośnik do okładki danego albumu. Opisane jest to w sekcji nr.~\ref{sec:dbrelations}. Warto wspomnieć, że albumy każdego autora muszą mieć unikalne nazwy - problem identyfikacji jest podobny jak przy autorach - lecz nazwy albumów różnych autorów mogą się powtarzać. 
	
	\item Piosenki - tabela ta zawiera informacje o wszystkich piosenkach w bibliotece, pozyskane z tagów plików.
\end{itemize}

Detale dotyczące każdej z tabel można przeczytać w sekcji nr.~\ref{sec:dbtables}.

\subsubsection{Opis pól tabel} \label{sec:dbtables}

\paragraph{Autorzy}

\begin{itemize}
	\item \textbf{nazwa} - unikalna nazwa autora, jest zarazem kluczem głównym
\end{itemize}

\paragraph{Albumy}

\begin{itemize}
	\item \textbf{id} - klucz główny, unikatowy identyfikator albumu
	
	\item \textbf{nazwa} - nazwa albumu, unikalna w obrębie jednego autora
	
	\item \textbf{tytuł} - tytuł albumu

	\item \textbf{okładka} - ścieżka do pliku z okładką albumu
\end{itemize}

\paragraph{Piosenki}

\begin{itemize}
	\item \textbf{id} - klucz główny, unikatowy identyfikator piosenki
	
	\item \textbf{tytuł} - tytuł piosenki
	
	\item \textbf{album} - klucz obcy, odniesienie do albumu, do którego należy piosenka
	
	\item \textbf{ścieżka} - ścieżka do pliku z piosenką
\end{itemize}

\subsubsection{Relacje w bazie} \label{sec:dbrelations}

\begin{figure}[H]
	\centering
	\includegraphics[width=1\textwidth]{images/ch3-db-relacje.drawio.png}
	\caption{\centering{Model relacji bazy}}
	\label{fig:dbrelations}
\end{figure}

Na rysunku nr.~\ref{fig:dbrelations} ukazany jest uproszczony model bazy danych, biorący tylko pod uwagę komponenty potrzebne do określenia relacji. Autorzy są w relacji $M$ do $N$ z albumami. Każdy autor może mieć wiele albumów, a każdy album wiele autorów. Piosenki z albumami są w relacji $1$ do $N$. Każda piosenka może należeć wyłącznie do jednego albumu, ale każdy album może mieć wiele piosenek.

Warto zauważyć, że piosenki niebezpośrednio łączą się z autorami. Jeżeli piosenka chce uzyskać swojego autora, musi zrobić to poprzez album.

\subsection{Czujnik światła}

Zadanie czujnika światła jest dość proste i niezbyt skomplikowane.
Celem czujnika jest wykrywanie intensywności światła i na podstawie tej intensywnośći aktualizacja wyglądu menu poprzez wykorzystanie schematów kolorów oraz motywów. Założenie jest takie, że przy intensywności światła mniejszej niż 50 procent z 10000 lumenów aplikacja będzie przełączać się na tryb ciemny, przy intensywności większej niż 50 procent aplikacja będzie przełączać się na tryb jasny.

\subsection{Uwierzytelnianie biometrią}
Aplikacja będzie wykorzystywać biometrię do autoryzacji użytkownika.
Przed wejściem do aplikacji zostanie wywołany ekran uwieżytelniania, na tym ekranie będzie okienko na którym będzie znajdowało się miejsce, gdzie będzie można przyłożyć palec. Aplikacja będzie sprawdzać czy odcisk palca znajduje się zapisany na telefonie i będzie wykorzystywać ten odcisk do uwierzytelniania.
