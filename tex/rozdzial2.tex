\newpage
\section{Określenie wymagań szczegółowych}		%2
%Dokładne określenie wymagań aplikacji (cel, zakres, dane wejściowe) – np. opisać przyciski, czujniki, wygląd layautu, wyświetlenie okienek. Opisać zachowanie aplikacji – co po kliknięciu, zdarzenia automatyczne. Opisać możliwość dalszego rozwoju oprogramowania. Opisać zachowania aplikacji w niepożądanych sytuacjach.

\subsection{Ogólny opis wymagań projektu}

Aplikacja jest zaprojektowana w Android Studio w języku Kotlin. Całe UI aplikacji będzie zbudowane na podstawie Frameworka \texttt{Jetpack Compose}\cite{doc_compose} Używając wbudowanych bibliotek w SDK Androida, będzie mogła odczytywać pliki ze wskazanego folderu. Odczytywanie tagów z plików odbędzie się za pomocą biblioteki \texttt{Media3}\cite{doc_media3}. Jest to oficjalna biblioteka Googlea do obsługi plików medialnych na Androidzie. Wszelki processing audio np. na potrzeby wizualizacji może zostać wykonany za pomocą SDK i wbudowanego modułu \texttt{AudioProcessor}\cite{doc_audioproc}. Odtwarzaniem pliku będzie zajmowała się biblioteka \texttt{ExoPlayer}\cite{doc_exoplayer}, posiadająca częściową integrację z \texttt{Media3}. Informacje o utworach powinny być ładowane do bazy danych. Będzie ona lokalnie, na urządzeniu. Ku temu celu, można użyć biblioteki \texttt{Room}\cite{doc_room}. Biblioteka ta jest wrapperem do wbudowanych funkcjonalności \texttt{SQL} Androida.

\subsection{Ogólny zarys narzędzi użytych w projekcie}

\subsubsection{Android Studio}

Android Studio jest IDE stworzonym przez Google, na bazie InteliJ IDEA od JetBrains. Jest ono przystosowane, jak z nazwy wynika, do tworzenia aplikacji na Androida. Ku temu celu posiada wiele udogodnień, odróżniających program od typowego edytora jak np. wbudowany emulator Androida, integrujący się z całym środowiskiem, czy preview różnych elementów interfejsu - gdzie elementy te generowane są w kodzie, a nie w osobnym języku jak np. xml - bez potrzeby dekompilacji całej aplikacji. 

Android Studio został użyty w projekcie, ponieważ:

\begin{itemize}
	\item Sam program jest crossplatformowy - nasz zespół używa wielu systemów operacyjnych. Platformy takie jak \texttt{MAUI}, są zespolone z Visual Stdio, czyli z Windowsem. Android Studio jest dostępny na wszystkie większe systemy operacyjne, co ułatwia nam pracę.
	\item Jest to program, zbudowany na podstawie IdeaJ, czyli zagłębiony jest w tym ekosystemie. Oznacza to dostęp do większej ilości pluginów niż np. Visual Studio, nie wspominając o ogólnej możliwości dostosowania ustawień.
\end{itemize}

Wady korzystania z Android Studio to m.in.

\begin{itemize}
	\item Duże wykorzystanie zasobów - program lubi zżerać duże ilości RAMu. W tym momencie, mając otwarty mały projekt + emulator, program wykorzystuje ponad 9GB RAMu. 
\end{itemize}

\subsubsection{Kotlin}

Kotlin został stworzony w 2010 roku przez firmę JetBrains oraz jest on przez nią rozwijany. Kotlin jest wieloplatformowym językiem typowanym statystycznie który został zaprojektowany aby współpracować z maszyną wirtualną Javy. Swoją nazwę zawdzięcza wyspie Kotlin która znajduje się w zatoce finlandzkiej.

Kotlin jest wykorzystywany w projekcie ze względów:

\begin{itemize}
	\item Jest on wspierany przez Android Studio, razem z Javą i C++. Kotlin ponadto, ma dostęp do nowoczesnych frameworków jak Jetpack Compose
	
	\item Jest on \textit{defakto} językiem do programowania na Androida - do niedawna Java mogła cieszyć się tym tytułem, ale od 2019 r. Google ogłosiło Kotlina jako rekomendowany język do tworzenia aplikacji na Android.
\end{itemize}

Składnia Kotlina wygląda następująco:

\begin{lstlisting}[caption=kotlin001 - Funkcje, label={lst:listing-k}, language=kotlin]
fun main() {
	printf("Czesc to ja, kotlin!")
}
\end{lstlisting}
Definicja funkcji wykonywana jest za pomocą "fun".

Zmienne w Kotlinie deklarowane są za pomocą \texttt{val} i \texttt{var}. Różnica polega na tym, że zmienne oznaczone \texttt{val} mogą zostać modyfikowane natomiast zmienne oznaczone \texttt{val} już nie.

\begin{lstlisting}[caption=kotlin002 - Zmienne, label={lst:listing-k}, language=kotlin]
	fun main() {
		var nazwa = "Projekt Android"
		val liczba = "777"
	}
\end{lstlisting}

Kompilator Kotlina posiada funkcję autodedukcji typów, więc w wielu wypadkach typu zmiennej nie trzeba adnotować.

\subsection{Wykorzystanie czujników}

\begin{itemize}
	\item Żyroskop - Z racji, że każdy element interfejsu w Jetpack jest generowany kodem, można, przynajmniej na początku, ustawić każdą wersję interfejsu jako osobną funkcję. Następnie, w zależności od wykrytej orientacji, przy użyciu API sensorów\cite{doc_sensorapi}, można wywoływać odpowiednią funkcję.
	
	\item Mikrofon - Funkcja dyktafonu najprawdopodobniej będzie całkiem oddzielnym Activity. Funkcjonalność ta, z natury, jest dosyć oddzielna od reszty aplikacji. Nagrania dyktafonem powinny być zapisywane do osobnego folderu. Można by zintegrować nagrania z resztą aplikacji jako osobnego wykonawcę w widoku biblioteki. Mikrofon będzie nagrywany poprzez moduł MediaRecorder\cite{doc_mediarecorder}

	\item Czujnik światła - Android Studio oferuje możliwość definiowania własnych klas zajmujących się kolorystyką. Oznacza to że można używać różnych obiektów w zależności od warunków. Wykrywanie światła będzie się odbywało używając API sensorów\cite{doc_sensorapi}
\end{itemize}


\subsection{Zachowanie w niepożądanych sytuacjach}

Głównym wyjątkiem, na który może napotkać się aplikacja jest błąd odczytu albo plików, albo tagów z pliku. Kotlin, na szczęście, pozwala na łatwe sprawdzanie wartości null danych zmiennych operatorem \texttt{?}. W odpowiednich fragmentach kodu dotyczących ładowania plików, będzie sprawdzana poprawność danych i najprawdopodobniej pojawi się pop-up po stronie użytkownika, że wystąpił błąd, a po stronie dewelopera błąd zostanie logowany.

\subsection{Dalszy rozwój}

Jeżeli praca nad aplikacją będzie się odbywała w przyszłości, należy skupić uwagę na lepszym zarządzaniu biblioteką (auto tagowanie, pobieranie miniatur z internetu, itp.). Ponadto, należy szukać błędów, które nadal zostały w aplikacji.
