	\newpage
\section{Implementacja}		%4
%Wkleić szkielet kodu, wraz z komentarzami. Opisać zmienne, struktury do czego służą. Opisać procedury, metody co wykonują. Opisać nowe zdefiniowane klasy. Opisać dziedziczenie. Opisać nowo utworzone pliki za co odpowiadają.

\subsection{Zarządzanie bazą danych}

\subsubsection{Klasa \texttt{DatabaseManager}} \label{sec:DatabaseManager}

Za zarządzanie bazą danych odpowiedzialna jest klasa \texttt{DatabaseManager}, której kod jest zamieszony na listingu nr. \ref{lst:DatabaseManager_struct}. Klasa jest wrapperem do bazy danych \texttt{Room}\cite{doc_room} i do niej akcesorów.

\begin{lstlisting}[caption=Strukutura klasy \texttt{DatabaseManager}, label={lst:DatabaseManager_struct}, language=kotlin]
@Singleton
class DatabaseManager @Inject constructor(
    @ApplicationContext context: Context
) {
    private val database: LibraryDb = Room.databaseBuilder(
        context,
        LibraryDb::class.java, "Library"
    ).build()

    fun collectAuthorsFlow(): Flow<List<Author>> = database.uiDao().getAllAuthorsFlow()

    fun collectAlbumsByAuthorFlow(authorName: String): Flow<List<Album>> {
        return database.uiDao().getAuthorWithAlbums(authorName)
            .map { it.albums }
    }

    fun collectSongsByAlbumFlow(albumId: Long): Flow<List<Song>> {
        return database.uiDao().getAlbumWithSongs(albumId)
            .map { it.songs }
    }
	
...

    fun populateDatabase(songs: List<TagExtractor.SongInfo>) {
        assert(Thread.currentThread().name != "main")

        val dao = database.logicDao()

        fun addAuthors() {
            songs.fastForEach { song ->
                //TODO: there should be a distinction between albumartists and regular artists
                song.albumArtists?.fastForEach { name ->
                    if(dao.getAuthor(name) == null) {
                        dao.insertAuthor(Author(name = name))
                    }

                }
            }
        }

        fun addAlbumsAndRelations() {
            // FIXME: xdddddddd
            val distinctAlbumArtistsList = songs
                .map { Triple(it.album, it.albumArtists, it.coverUri) }
                .distinct()
            Log.d(javaClass.simpleName, "Distinct artists set: $distinctAlbumArtistsList")

            distinctAlbumArtistsList.fastForEach {
                val albumTitle = it.first.toString()
                val artists = it.second
                val coverUri = it.third

                val albumId = dao.insertAlbum(Album(
                    title = albumTitle,
                    coverUri = coverUri.toString(),
                ))

                artists?.fastForEach {
                    dao.insertAlbumAuthorCrossRef(AlbumAuthorCrossRef(
                        albumId = albumId,
                        name = it.toString()
                    ))
                }
            }
        }

        fun addSongs() {
            songs.fastForEach { song ->
                Log.d(javaClass.simpleName, "NEW SONG\n")
                Log.d(javaClass.simpleName, "Album artists: ${song.albumArtists}")

                val albumWithAuthorCandidates = dao
                    .getAlbumsByTitle(song.album.toString())
                    .map { it.albumId }
                    .map { dao.getAlbumWithAuthors(it) }
                Log.d(javaClass.simpleName, "$albumWithAuthorCandidates")

                var correctAlbum: Album? = null
                albumWithAuthorCandidates.fastForEach {
                    Log.d(javaClass.simpleName, "${song.albumArtists}, ${it.authors}")
                    //FIXME: theese guys shouldn't be ordered, will have to refactor a bunch of
                    // stuff with sets instead of lists
                    if(song.albumArtists?.sorted() == it.authors.map { it.name }.sorted()) {
                        correctAlbum = it.album
                    }
                }

                dao.insertSong(Song(
                    title = song.title,
                    albumId = correctAlbum?.albumId,
                    fileUri = song.fileUri.toString(),
                ))
            }
        }

        addAuthors()
        addAlbumsAndRelations()
        addSongs()
    }
}
\end{lstlisting}

% TODO: opisac co to hilt
Na pierwszej linijce można zauważyć adnotację \texttt{@Singleton}. Pochodzi ona z bilbioteki \texttt{Hilt}\cite{doc_hilt}. Powiadamia ona bibliotekę o tym że klasa jest singletonem, czyli że ma istnieć tylko jej jedna instancja na cały program. Uczyniono to, dlatego że baza danych powinna być jedna na całą aplikację. Menadżer z nią interfejsujący, dlatego że jest używany w wielu innych klasach, też powinien mieć tylko jedną instancję, aby nie marnować pamięci.

Na linijce nr. 2, widać konstruktor klasy, do którego też przy użyciu \texttt{Hilt}, wstrzykiwany jest \texttt{context}.

Następnie, na linijce nr. 5, widać inicjalizację samego obiektu bazy \texttt{database}. Baza jest reprezentowana przez klasę \texttt{LibraryDb}, definicję której można zobaczyć w sekcji \ref{sec:LibraryDb}

% TODO: opisac co to flow albo w 2 chapterze albo 3
Dalej, do linijki nr. 22 pokazane są metody zwracające rózne elementy bazy. Wiekszosc z tych metod zwraca \texttt{Flow}\cite{TODO:}. \texttt{Room} natywnie obsługuje \texttt{Flowy}, a dlatego że wymusza dostęp do bazy z innych wątków niż główny, większość operacji wykonywanych na bazie odbywa się za pośrednictwem typów \texttt{Flow}

Same metody są wrapperami do obiektów \texttt{Dao} bazy. Więcej o nich w sekcji \ref{sec:daos}. Niektóre obrabiają dane jak np. \texttt{collectSongsByAlbumFlow()} na linijce nr. 17., która mapuje zwraca piosenki z wyjściowej klasy relacyjnej.

Metod tych jest więcej, lecz wyglądają one bardzo podobnie. Dla zwięzłości, mozna je pominąć.

%TODO: TagExtractor.SongInfo to będzie po prostu SongInfo jak sie mi zeche w kodzie zmienic
Metoda \texttt{populateDatabase()} zadeklarowana na linijce nr. 24, jest odpowiedzialna za ładowanie wyjętych z plików informacji do bazy. Jako parametr odstaje ona zmienną \texttt{songs} typu \texttt{List<TagExtractor.SongInfo>}   Zadeklarowane są w niej trzy funkcje pomocnicze: \texttt{addAuthors()}, \texttt{addAlbumsAndRelations()} i \texttt{addSongs()}. Wywoływane są one po kolei w metodzie głównej.

Funkcja \texttt{addAuthors()}, zadeklarowana na linijce nr. 29, jest prosta w swoim działaniu. Lista z \texttt{SongInfo} jest iterowana i po kolei wpisywani są wszyscy autorzy, którzy jeszcze w bazie nie istnieją.

Funkcja \texttt{addAlbumsAndRelations()}, zadeklarowana na linijce nr. 41, odpowiada za dodawanie albumów do bazy oraz tworzenie relacji między nimi, a autorami. Tworzona zmienna \texttt{distinctAlbumArtistsList} mapuje tylko unikalne pary albumów i autorów (zmienna \texttt{coverUri} nie ma znaczenia przy określaniu autorstwa, jest przypisywana tutaj dlatego, że trudno było znaleźć dla niej lepsze miejsce). Dzięki temu początkowemu filtrowaniu, wiadomo, że każdy napotkany album będzie unikalny. Następnie, \texttt{distinctAlbumArtistsList} jest iterowana - przy każdej iteracji dodawany jest nowy album do bazy. Metoda \texttt{insertAlbum()} zwraca \texttt{id} nowo dodanego albumu. Wynik jej jest przypisywany do zmiennej \texttt{albumId} na linijce nr. 53. Potem, zostaje przypisywana relacja albumu z autorami. Autorów może być kilku, więc są oni reprezentowani przy każdej iteracji przez listę, która jest iterowana, a relacja zostaje dodawana z nazwą autora i \texttt{albumId}.

Funkcja \texttt{addSongs()}, zadeklarowana na linijce nr. 67, ma na celu dodanie piosenek do bazy. Ciało funkcji jest w pętli iterującej się przez listę piosenek. Na początku pętli, na linijce nr. 72 deklarowana jest zmienna \texttt{albumWithAuthorCandidates}. Jest ona listą relacji album - autorzy wszystkich albumów o tej samej nazwie co ten w danym elemencie listy. Następnie, lista ta jest iterowana i przy każdej iteracji sprawdzane jest czy lista autorów w danej relacji jest równa z listą autorów danej piosenki. Jeżeli tak, wartość danego albumu z wybranej relacji jest przypisywana do zmiennej zadeklarowanej na lini nr. 78 \texttt{correctAlbum}. Na końcu funkcji, piosenka dodana jest do bazy przy użyciu metody \texttt{dao}.

\subsubsection{Klasa LibraryDb} \label{sec:LibraryDb}

Klasa \texttt{LibraryDb} jest deklaracją faktycznej instancji bazy danych, która jest implementowana i generowana przez bibliotekę \texttt{Room}. Z racji tego, że jest to klasa abstrakcyjna, jej zadaniem jest określenie struktury bazy i jakie komponenty ma ona zawierać

\begin{lstlisting}[caption=Deklaracja bazy LibraryDb, label={lst:LibraryDb_class}, language=kotlin]
@Database(entities = [Song::class, Author::class, Album::class, AlbumAuthorCrossRef::class], version
= 1)
abstract class LibraryDb : RoomDatabase() {
    abstract fun logicDao(): LogicDao
    abstract fun uiDao(): UIDao
}

\end{lstlisting}

Jak widać na listingu nr.~\ref{lst:LibraryDb_class}, na początku klasy należy zamieścić adnotację \texttt{@Database}. Powiadamia ona bibliotekę \texttt{Room} o tym, że następująca klasa jest bazą danych. Parametr \texttt{entities} określa wszystkie tabele jakie mają się w klasie zawierać. O tabelach więcej w sekcji nr.~\ref{sec:tables}. Parametr \texttt{version} zajmuje się wersjonowaniem bazy. Jest on ważny przy aktualizacjach aplikacji, aby baza mogła zostać odpowiednio zmieniona. 

Na linijce nr. 3 umieszczona jest faktyczna deklaracja klasy. Dziedziczy ona z klasy \texttt{RoomDatabase}. Jedyne rzeczy jakie są do dziecięcej klasy dodawane, to metody zwracające obiekty \texttt{dao}, opisane w sekcji nr.~\ref{sec:daos}.

%TODO: 
\subsubsection{Obiekty Dao} \label{sec:daos}

\subsubsection{Tabele} \label{sec:tables}

W bibliotece \texttt{Room}, każda tabela to \texttt{dataclass} określana adnotacją \texttt{@Entity}. Kolumny takiej tabeli to po prostu pola klasy. Klucz danej tabeli jest określany adnotacją \texttt{@PrimaryKey}

\paragraph{Author}

\begin{lstlisting}[caption=Deklaracja tabeli Author, label={lst:Author_class}, language=kotlin]
@Entity
data class Author (
    @PrimaryKey val name: String
)
\end{lstlisting}

Tabela \texttt{Author}, zawarta na listingu nr.~\ref{lst:Author_class}, określa autorów. Tabela jest prosta, jedynym polem jest \texttt{name}, który jest kluczem.

\paragraph{Album}

\begin{lstlisting}[caption=Deklaracja tabeli Album, label={lst:Album_class}, language=kotlin]
@Entity
data class Album(
    @PrimaryKey(autoGenerate = true) val albumId: Long = 0,
    val title: String,
    val coverUri: String?,
)
\end{lstlisting}

Tabela \texttt{Album}, zawarta na listingu nr.~\ref{lst:Album_class}, określa albumy. Kluczem jest zmienna \texttt{albumId}. Klucz jest generowany automatycznie, dzięki parametrowi adnotacji \texttt{autoGenerate}. Pole \texttt{title} określa tytuł, a pole \texttt{coverUri} określa adres \texttt{URI} okładki.

\paragraph{Song}

\begin{lstlisting}[caption=Deklaracja tabeli Song, label={lst:Song_class}, language=kotlin]

@Entity(
    foreignKeys = [
        ForeignKey(
            entity = Album::class,
            parentColumns = ["albumId"],
            childColumns = ["albumId"],
            onDelete = ForeignKey.CASCADE
        )
    ],

    indices = [Index(value = ["albumId"])]
)
data class Song(
    @PrimaryKey(autoGenerate = true) val songId: Long = 0,
    val title: String?,
    val albumId: Long?,
    val fileUri: String?,
)

\end{lstlisting}

Klasa ta, zawarta na listingu nr.~\ref{lst:Song_class}, określa tabelę piosenek. Pole \texttt{foreignKeys} w adnotacji \texttt{@Entity} określa obce klucze, którymi posługuje się klasa. W tym przypadku określone jest to, że pole w \texttt{Song} \texttt{albumId} wskazuje na pole w \texttt{Album} \texttt{albumId} Pole \texttt{indices} każe indeksować pola z \texttt{albumId} ku polepszeniu szybkości bazy. W ciele klasy, pole \texttt{title} to tytuł piosenki. Pole \texttt{albumId} określa ID albumu, do którego należy dana piosenka. 

\subsubsection{Relacje} \label{sec:relations}

Relacje w Room są określane jako osobne \texttt{dataclassy}. Są one zadeklarowane adnotacją \texttt{@Relation} w danej klasie. Ponadto umieszczenie elementu w adnotacji \texttt{@Embedded}, pozwala klasie \enquote{przyswoić} pola danego elementu. Dzięki temu klasa może odnosić się do pól danego elementu tak jakby były one bezpośrednio w klasie. Konieczne jest umieszczenie elementu głównego, od którego będzie relacja wychodziła, do tej adnotacji.

\paragraph{AlbumWithSongs} \label{sec:albumwithsongs} \

Klasa \texttt{AlbumWithSongs} na listingu nr.~\ref{lst:AlbumWithSongs_class}, określa relację albumów i piosenek. 

\begin{lstlisting}[caption=Deklaracja relacji \texttt{AlbumWithSongs}, label={lst:AlbumWithSongs_class}, language=kotlin]
data class AlbumWithSongs(
    @Embedded val album: Album,
    @Relation(
        parentColumn = "albumId",
        entityColumn = "albumId"
    )
    val songs: List<Song>
)
\end{lstlisting}

Relacja łączy pole \texttt{albumId} albumu z polem \texttt{albumId} piosenek. Pole \texttt{songs} zawiera wszystkie piosenki z tą samą wartością pola \texttt{albumId} co faktyczny klucz danego albumu.

\paragraph{AlbumAuthorCrossRef} \label{sec:AlbumAuthorCrossRef}

\begin{lstlisting}[caption=Deklaracja tabeli relacji \texttt{AlbumAuthorCrossRef}, label={lst:AlbumAuthorCrossRef_class}, language=kotlin]
@Entity(primaryKeys = ["albumId", "name"])
data class AlbumAuthorCrossRef(
    val albumId: Long,
    val name: String
)

\end{lstlisting}

Tabela na listingu nr.~\ref{lst:AlbumAuthorCrossRef_class} określa relację $M$ do $N$ między albumami a autorami. Jest to tabela z dwoma kluczami głównymi: \texttt{albumId} dla tabeli \texttt{Album} i \texttt{name} dla tabeli \texttt{Author}.

\paragraph{AlbumWithAuthors i AuthorWithAlbums} \

Obie klasy są do siebie bardzo podobne więc zostaną omówione razem. 

\begin{lstlisting}[caption=Deklaracja relacji \texttt{AlbumWithAuthors}, label={lst:AlbumWithAuthors_class}, language=kotlin]
data class AlbumWithAuthors(
    @Embedded val album: Album,
    @Relation(
        parentColumn = "albumId",
        entityColumn = "name",
        associateBy = Junction(AlbumAuthorCrossRef::class)
    )
    val authors: List<Author>
)
\end{lstlisting}

\begin{lstlisting}[caption=Deklaracja relacji \texttt{AuthorWithAlbums}, label={lst:AuthorWithAlbums_class}, language=kotlin]
data class AuthorWithAlbums(
    @Embedded val author: Author,
    @Relation(
        parentColumn = "name",
        entityColumn = "albumId",
        associateBy = Junction(AlbumAuthorCrossRef::class)
    )
    val albums: List<Album>
)
\end{lstlisting}

Na listingu nr.~\ref{lst:AlbumWithAuthors_class} przedstawiona jest klasa \texttt{AlbumWithAuthors}. Definiuje ona relację danego albumu z jego autorami. Dlatego, że relacja jest $M$ do $N$, w adnotacji \texttt{@Relation} dodane jest odniesienie do tabeli relacji \texttt{AlbumAuthorCrossRef}, opisanej w sekcji nr. ~\ref{sec:AlbumAuthorCrossRef}. Klucze, jakie mają być porównywane są zdefiniowane w parametrach \texttt{parentColumn}, dla \texttt{id} albumu i \texttt{entityColumn} dla nazwy autora. Wynikiem tej relacji jest lista albumów. Sytuacja wygląda podobnie w \texttt{AuthorWithAlbums}, na listingu nr.~\ref{lst:AuthorWithAlbums_class}. Tym razem to autor jest rodzicem i oczekujemy od relacji listy albumów danego autora.

\subsection{Czujnik światła}
Czujnik światła został zaimplementowany za pomocą wbudowanej funkcji. Zadaniem czujnika jest dynmiczna zmiana schematu kolorów aplikacji na podstawie danych otrzymanych z czujnika światła wbudowanego w urządzeniu mobilnym z systemem android.

\begin{lstlisting}[caption=Implementacja czujnika światłą w \texttt{MainActivity.kt}, label={lst:LightSensorMain}, language=kotlin]
	@AndroidEntryPoint
	class MainActivity : FragmentActivity(), SensorEventListener {
		private lateinit var sensorManager: SensorManager
		private var lightSensor: Sensor? = null
		private val _isDarkTheme = mutableStateOf(false)
		private val isDarkTheme: State<Boolean> = _isDarkTheme
		
		private val _isAuthenticated = mutableStateOf(false)
		private val isAuthenticated: State<Boolean> = _isAuthenticated
		
		override fun onCreate(savedInstanceState: Bundle?) {
			super.onCreate(savedInstanceState)
			val biometricAuthenticator = BiometricAuthenticator(this)
			
			sensorManager = getSystemService(Context.SENSOR_SERVICE) as SensorManager
			lightSensor = sensorManager.getDefaultSensor(Sensor.TYPE_LIGHT)
			
			enableEdgeToEdge()
			
			setContent {
				val darkTheme by isDarkTheme
				val authenticated by isAuthenticated
				RaptorTheme(darkTheme = darkTheme) {
					Surface(
					modifier = Modifier.fillMaxSize(),
					color = MaterialTheme.colorScheme.background
					) {
						if (authenticated) {
							MainScreen()
						} else {
							AuthenticationScreen(
							onAuthenticate = {
								promptBiometricAuthentication(biometricAuthenticator)
							}
							)
						}
					}
				}
			}
		}
		
		private fun promptBiometricAuthentication(biometricAuthenticator: BiometricAuthenticator) {
			biometricAuthenticator.PromptBiometricAuth(
			title = "Authentication Required",
			subtitle = "Please authenticate to proceed",
			negativeButtonText = "Cancel",
			fragmentActivity = this,
			onSuccess = {
				runOnUiThread {
					_isAuthenticated.value = true
				}
			},
			onFailed = {
			},
			onError = { errorCode, errorString ->
			}
			)
		}
		
		override fun onResume() {
			super.onResume()
			lightSensor?.let { sensor ->
				sensorManager.registerListener(this, sensor, SensorManager.SENSOR_DELAY_NORMAL)
			}
		}
		
		override fun onPause() {
			super.onPause()
			sensorManager.unregisterListener(this)
		}
		
		override fun onSensorChanged(event: SensorEvent?) {
			if (event?.sensor?.type == Sensor.TYPE_LIGHT) {
				val lightLevel = event.values[0]
				val maxLightLevel = lightSensor?.maximumRange ?: 10000f
				
				_isDarkTheme.value = lightLevel < 0.4 * maxLightLevel
			}
		}
		
		override fun onAccuracyChanged(sensor: Sensor?, accuracy: Int) {
		}
	}
\end{lstlisting}

Na listingu \ref{lst:LightSensorMain} przedstawiona jest funkcja MainScreen. W wierszu 2 mamy SensorEventListener, który jest frameworkiem w androidzie który pozwala aplikacji na reagowanie na zmiany odczytywane przez czujniki smartfona. Po dodaniu automatycznie tworzone są funkcje onSensorChanged, onResume, onPause, onAccuracyChanged.
Implementację sensora zaczynamy od utworzenia zmiennych sensormanager oraz lightSensor w wierszach 3 i 4. Zmienna sensormanager odpowiedzialna jest za pobranie menadżera czujników z poziomu systemu.
Zmienna \texttt{lightSensor} odpowiedzialna jest za uzyskanie referencji do głównego czujnika urządzenia, jeżeli się nie uda to zwraca wartość null.
Zmienna \texttt{\_isdarkTheme} w wierszu 5 określa czy aplikacja wykorzystuje obecnie tryb ciemny, zmienna \texttt{isDarkTheme} w wierszu 6 pomaga jej w tym za pomocą odczytu w UI.
\\
W \texttt{SetContent} w wierszu 23 do dynamicznej zmiany kolorów wykorzytywane jest "RaptorTheme(darkTheme = darmTheme)" zdefiniowany w \texttt{Theme.kt} w folderze \texttt{ui.Theme}.
\\
Poniżej, w wierszach od 60 do 65 znajduje się funkcja \texttt{onResume}, która rejestruje słuchacza zdarzeń po wznowieniu działania aplikacji. odpowiedzialne jest za częstotliwość aktualizacji(\texttt{SENSOR\_DELAY\_NORMAL}).
\\
This oznacza implementację \texttt{SensorEventListener}.
\\
Funkcja \texttt{onpause} odpowiedzialna jest zatrzymanie działania słuchacza zdarzeń w przypadku pracy aplikacji w tle.
\\
Funkcja \texttt{onSensorChanged} wywoływana jest za każdym razem gdy uzyskany zostanie nowy odczyt z czujnika systemowego.
Zmienna \texttt{lightLevel} pobiera aktualny poziom oświetlenią(jednosta to luks).
Zmienna \texttt{maxLightLevel} pobiera maksymalny zakres pomiarowy czujnika. W przypadku braku przypisana zostanie wartość 10000 luksów. 
W wierszu 77 znajduje się instrukcja przejścia w tryb ciemny jeżeli obecny wykrywany poziom światła jest mniejszy niż 40 procent.
Funkcja \texttt{onAccuracyChanged} nie jest tutaj wykorzystywana. Może być wykorzystana do np. zmiany dokładności wykrywania czujnika.
\\


\subsection{Autoryzacja odciskiem palca}

\subsection{Sound wave}
