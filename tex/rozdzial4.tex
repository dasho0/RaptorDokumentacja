	\newpage
\section{Implementacja}		%4
%Wkleić szkielet kodu, wraz z komentarzami. Opisać zmienne, struktury do czego służą. Opisać procedury, metody co wykonują. Opisać nowe zdefiniowane klasy. Opisać dziedziczenie. Opisać nowo utworzone pliki za co odpowiadają.

\subsection{Zarządzanie bazą danych}

\subsubsection{Klasa \texttt{DatabaseManager}} \label{sec:DatabaseManager}

Za zarządzanie bazą danych odpowiedzialna jest klasa \texttt{DatabaseManager}, której kod jest zamieszony na listingu nr. \ref{lst:DatabaseManager_struct}. Klasa jest wrapperem do bazy danych \texttt{Room}\cite{doc_room} i do niej akcesorów.

\begin{lstlisting}[caption=Strukutura klasy \texttt{DatabaseManager}, label={lst:DatabaseManager_struct}, language=kotlin]
@Singleton
class DatabaseManager @Inject constructor(
    @ApplicationContext context: Context
) {
    private val database: LibraryDb = Room.databaseBuilder(
        context,
        LibraryDb::class.java, "Library"
    ).build()

    fun collectAuthorsFlow(): Flow<List<Author>> = database.uiDao().getAllAuthorsFlow()

    fun collectAlbumsByAuthorFlow(authorName: String): Flow<List<Album>> {
        return database.uiDao().getAuthorWithAlbums(authorName)
            .map { it.albums }
    }

    fun collectSongsByAlbumFlow(albumId: Long): Flow<List<Song>> {
        return database.uiDao().getAlbumWithSongs(albumId)
            .map { it.songs }
    }
	
...

    fun populateDatabase(songs: List<TagExtractor.SongInfo>) {
        assert(Thread.currentThread().name != "main")

        val dao = database.logicDao()

        fun addAuthors() {
            songs.fastForEach { song ->
                //TODO: there should be a distinction between albumartists and regular artists
                song.albumArtists?.fastForEach { name ->
                    if(dao.getAuthor(name) == null) {
                        dao.insertAuthor(Author(name = name))
                    }

                }
            }
        }

        fun addAlbumsAndRelations() {
            // FIXME: xdddddddd
            val distinctAlbumArtistsList = songs
                .map { Triple(it.album, it.albumArtists, it.coverUri) }
                .distinct()
            Log.d(javaClass.simpleName, "Distinct artists set: $distinctAlbumArtistsList")

            distinctAlbumArtistsList.fastForEach {
                val albumTitle = it.first.toString()
                val artists = it.second
                val coverUri = it.third

                val albumId = dao.insertAlbum(Album(
                    title = albumTitle,
                    coverUri = coverUri.toString(),
                ))

                artists?.fastForEach {
                    dao.insertAlbumAuthorCrossRef(AlbumAuthorCrossRef(
                        albumId = albumId,
                        name = it.toString()
                    ))
                }
            }
        }

        fun addSongs() {
            songs.fastForEach { song ->
                Log.d(javaClass.simpleName, "NEW SONG\n")
                Log.d(javaClass.simpleName, "Album artists: ${song.albumArtists}")

                val albumWithAuthorCandidates = dao
                    .getAlbumsByTitle(song.album.toString())
                    .map { it.albumId }
                    .map { dao.getAlbumWithAuthors(it) }
                Log.d(javaClass.simpleName, "$albumWithAuthorCandidates")

                var correctAlbum: Album? = null
                albumWithAuthorCandidates.fastForEach {
                    Log.d(javaClass.simpleName, "${song.albumArtists}, ${it.authors}")
                    //FIXME: theese guys shouldn't be ordered, will have to refactor a bunch of
                    // stuff with sets instead of lists
                    if(song.albumArtists?.sorted() == it.authors.map { it.name }.sorted()) {
                        correctAlbum = it.album
                    }
                }

                dao.insertSong(Song(
                    title = song.title,
                    albumId = correctAlbum?.albumId,
                    fileUri = song.fileUri.toString(),
                ))
            }
        }

        addAuthors()
        addAlbumsAndRelations()
        addSongs()
    }
}
\end{lstlisting}

% TODO: opisac co to hilt
Na pierwszej linijce można zauważyć adnotację \texttt{@Singleton}. Pochodzi ona z bilbioteki \texttt{Hilt}\cite{doc_hilt}. Powiadamia ona bibliotekę o tym że klasa jest singletonem, czyli że ma istnieć tylko jej jedna instancja na cały program. Uczyniono to, dlatego że baza danych powinna być jedna na całą aplikację. Menadżer z nią interfejsujący, dlatego że jest używany w wielu innych klasach, też powinien mieć tylko jedną instancję, aby nie marnować pamięci.

Na linijce nr. 2, widać konstruktor klasy, do którego też przy użyciu \texttt{Hilt}, wstrzykiwany jest \texttt{context}.

Następnie, na linijce nr. 5, widać inicjalizację samego obiektu bazy \texttt{database}. Baza jest reprezentowana przez klasę \texttt{LibraryDb}, definicję której można zobaczyć w sekcji \ref{sec:LibraryDb}

% TODO: opisac co to flow albo w 2 chapterze albo 3
Dalej, do linijki nr. 22 pokazane są metody zwracające rózne elementy bazy. Wiekszosc z tych metod zwraca \texttt{Flow}\cite{TODO:}. \texttt{Room} natywnie obsługuje \texttt{Flowy}, a dlatego że wymusza dostęp do bazy z innych wątków niż główny, większość operacji wykonywanych na bazie odbywa się za pośrednictwem typów \texttt{Flow}

Same metody są wrapperami do obiektów \texttt{Dao} bazy. Więcej o nich w sekcji \ref{sec:daos}. Niektóre obrabiają dane jak np. \texttt{collectSongsByAlbumFlow()} na linijce nr. 17., która mapuje zwraca piosenki z wyjściowej klasy relacyjnej.

Metod tych jest więcej, lecz wyglądają one bardzo podobnie. Dla zwięzłości, mozna je pominąć.

%TODO: TagExtractor.SongInfo to będzie po prostu SongInfo jak sie mi zeche w kodzie zmienic
Metoda \texttt{populateDatabase()} zadeklarowana na linijce nr. 24, jest odpowiedzialna za ładowanie wyjętych z plików informacji do bazy. Jako parametr odstaje ona zmienną \texttt{songs} typu \texttt{List<TagExtractor.SongInfo>}   Zadeklarowane są w niej trzy funkcje pomocnicze: \texttt{addAuthors()}, \texttt{addAlbumsAndRelations()} i \texttt{addSongs()}. Wywoływane są one po kolei w metodzie głównej.

Funkcja \texttt{addAuthors()}, zadeklarowana na linijce nr. 29, jest prosta w swoim działaniu. Lista z \texttt{SongInfo} jest iterowana i po kolei wpisywani są wszyscy autorzy, którzy jeszcze w bazie nie istnieją.

Funkcja \texttt{addAlbumsAndRelations()}, zadeklarowana na linijce nr. 41, odpowiada za dodawanie albumów do bazy oraz tworzenie relacji między nimi, a autorami. Tworzona zmienna \texttt{distinctAlbumArtistsList} mapuje tylko unikalne pary albumów i autorów (zmienna \texttt{coverUri} nie ma znaczenia przy określaniu autorstwa, jest przypisywana tutaj dlatego, że trudno było znaleźć dla niej lepsze miejsce). Dzięki temu początkowemu filtrowaniu, wiadomo, że każdy napotkany album będzie unikalny. Następnie, \texttt{distinctAlbumArtistsList} jest iterowana - przy każdej iteracji dodawany jest nowy album do bazy. Metoda \texttt{insertAlbum()} zwraca \texttt{id} nowo dodanego albumu. Wynik jej jest przypisywany do zmiennej \texttt{albumId} na linijce nr. 53. Potem, zostaje przypisywana relacja albumu z autorami. Autorów może być kilku, więc są oni reprezentowani przy każdej iteracji przez listę, która jest iterowana, a relacja zostaje dodawana z nazwą autora i \texttt{albumId}.

%TODO: opisac insertSong()

%TODO:
\subsubsection{Klasa LibraryDb} \label{sec:LibraryDb}

%TODO: 
\subsubsection{Obiekty Dao} \label{sec:daos}
