	\newpage
\section{Implementacja}		%4
%Wkleić szkielet kodu, wraz z komentarzami. Opisać zmienne, struktury do czego służą. Opisać procedury, metody co wykonują. Opisać nowe zdefiniowane klasy. Opisać dziedziczenie. Opisać nowo utworzone pliki za co odpowiadają.

\subsection{Zarządzanie bazą danych}

\subsubsection{Klasa \texttt{DatabaseManager}} \label{sec:DatabaseManager}

Za zarządzanie bazą danych odpowiedzialna jest klasa \texttt{DatabaseManager}, której kod jest zamieszony na listingu nr. \ref{lst:DatabaseManager_struct}. Klasa jest wrapperem do bazy danych \texttt{Room}\cite{doc_room} i do niej akcesorów.

\begin{lstlisting}[caption=Strukutura klasy \texttt{DatabaseManager}, label={lst:DatabaseManager_struct}, language=kotlin]
@Singleton
class DatabaseManager @Inject constructor(
    @ApplicationContext context: Context
) {
    private val database: LibraryDb = Room.databaseBuilder(
        context,
        LibraryDb::class.java, "Library"
    ).build()

    fun collectAuthorsFlow(): Flow<List<Author>> = database.uiDao().getAllAuthorsFlow()

    fun collectAlbumsByAuthorFlow(authorName: String): Flow<List<Album>> {
        return database.uiDao().getAuthorWithAlbums(authorName)
            .map { it.albums }
    }

    fun collectSongsByAlbumFlow(albumId: Long): Flow<List<Song>> {
        return database.uiDao().getAlbumWithSongs(albumId)
            .map { it.songs }
    }
	
...

    fun populateDatabase(songs: List<TagExtractor.SongInfo>) {
        assert(Thread.currentThread().name != "main")

        val dao = database.logicDao()

        fun addAuthors() {
            songs.fastForEach { song ->
                //TODO: there should be a distinction between albumartists and regular artists
                song.albumArtists?.fastForEach { name ->
                    if(dao.getAuthor(name) == null) {
                        dao.insertAuthor(Author(name = name))
                    }

                }
            }
        }

        fun addAlbumsAndRelations() {
            // FIXME: xdddddddd
            val distinctAlbumArtistsList = songs
                .map { Triple(it.album, it.albumArtists, it.coverUri) }
                .distinct()
            Log.d(javaClass.simpleName, "Distinct artists set: $distinctAlbumArtistsList")

            distinctAlbumArtistsList.fastForEach {
                val albumTitle = it.first.toString()
                val artists = it.second
                val coverUri = it.third

                val albumId = dao.insertAlbum(Album(
                    title = albumTitle,
                    coverUri = coverUri.toString(),
                ))

                artists?.fastForEach {
                    dao.insertAlbumAuthorCrossRef(AlbumAuthorCrossRef(
                        albumId = albumId,
                        name = it.toString()
                    ))
                }
            }
        }

        fun addSongs() {
            songs.fastForEach { song ->
                Log.d(javaClass.simpleName, "NEW SONG\n")
                Log.d(javaClass.simpleName, "Album artists: ${song.albumArtists}")

                val albumWithAuthorCandidates = dao
                    .getAlbumsByTitle(song.album.toString())
                    .map { it.albumId }
                    .map { dao.getAlbumWithAuthors(it) }
                Log.d(javaClass.simpleName, "$albumWithAuthorCandidates")

                var correctAlbum: Album? = null
                albumWithAuthorCandidates.fastForEach {
                    Log.d(javaClass.simpleName, "${song.albumArtists}, ${it.authors}")
                    //FIXME: theese guys shouldn't be ordered, will have to refactor a bunch of
                    // stuff with sets instead of lists
                    if(song.albumArtists?.sorted() == it.authors.map { it.name }.sorted()) {
                        correctAlbum = it.album
                    }
                }

                dao.insertSong(Song(
                    title = song.title,
                    albumId = correctAlbum?.albumId,
                    fileUri = song.fileUri.toString(),
                ))
            }
        }

        addAuthors()
        addAlbumsAndRelations()
        addSongs()
    }
}
\end{lstlisting}

% TODO: opisac co to hilt
Na pierwszej linijce można zauważyć adnotację \texttt{@Singleton}. Pochodzi ona z bilbioteki \texttt{Hilt}\cite{doc_hilt}. Powiadamia ona bibliotekę o tym że klasa jest singletonem, czyli że ma istnieć tylko jej jedna instancja na cały program. Uczyniono to, dlatego że baza danych powinna być jedna na całą aplikację. Menadżer z nią interfejsujący, dlatego że jest używany w wielu innych klasach, też powinien mieć tylko jedną instancję, aby nie marnować pamięci.

Na linijce nr. 2, widać konstruktor klasy, do którego też przy użyciu \texttt{Hilt}, wstrzykiwany jest \texttt{context}.

Następnie, na linijce nr. 5, widać inicjalizację samego obiektu bazy \texttt{database}. Baza jest reprezentowana przez klasę \texttt{LibraryDb}, definicję której można zobaczyć w sekcji \ref{sec:LibraryDb}

% TODO: opisac co to flow
Dalej, do linijki nr. 22 pokazane są metody zwracające rózne elementy bazy. Wiekszosc z tych metod zwraca \texttt{Flow}\cite{TODO:}. \texttt{Room} natywnie obsługuje \texttt{Flowy}, a dlatego że wymusza dostęp do bazy z innych wątków niż główny, większość operacji wykonywanych na bazie odbywa się za pośrednictwem typów \texttt{Flow}

Same metody są wrapperami do obiektów \texttt{Dao} bazy. Więcej o nich w sekcji \ref{sec:daos}. Niektóre obrabiają dane jak np. \texttt{collectSongsByAlbumFlow()} na linijce nr. 17., która mapuje zwraca piosenki z wyjściowej klasy relacyjnej.

Metod tych jest więcej, lecz wyglądają one bardzo podobnie. Dla zwięzłości, mozna je pominac.

% TODO: opis populateDatabase()

\subsubsection{Klasa LibraryDb} \label{sec:LibraryDb}

\subsubsection{Obiekty Dao} \label{sec:daos}
