	\newpage
\section{Ogólne określenie wymagań projektu}		%1
%Ogólne określenie wymagań i zakresu programu (Czyli zleceniodawca określa wymagania programu) 

\subsection{Ogólny zarys wymagań}

Celem programu jest pełnienie funkcji odtwarzacza muzyki. Program będzie mógł skanować dany folder i jego podfoldery, a w nich zawarty pliki muzyczne i tworzyć na ich podstawie bibliotekę, zapisaną na dysku.  

\subsection{Wykorzystane czujniki}

Program ma na celu wykorzystanie trzech czujników, z którymi użytkownik będzie wchodził w interakcję. Zostaną użyte następujące:

\begin{itemize}
	\item Żyroskop - Interfejs programu będzie się zmieniał w zależności od orientacji urządzenia. 
	
	\item Wykrywacz odcisków palca - dostęp do programu powinien być ograniczony dla użytkowników mogących zweryfikować swój odcisk.

	\item Czujnik światła - Interfejs programu będzie mógł zmieniać swoje kolory w zależności od wykrytego poziomu światła na czujniku
\end{itemize}

\subsection{Zarys interfejsu}

\todo{Zmienić mockup widoku artystów. Dodać topbar i dać pola tekstowe w kafelki}

\begin{figure}[H]
	\centering
	\includegraphics[width=1\linewidth]{images/mockup2_artysta}
	\caption{\centering{Mockup widoku biblioteki - listing wykonawców}}
	\label{fig:mockup2artysta}
\end{figure}

Widok wykonawców, jest przedstawiony na rysunku nr.~\ref{fig:mockup2artysta}. Ten widok będzie ekranem startowym aplikacji. Jako "Kafelki", zwracać będzie się dokument do ułożonych równomiernie na rysunku kwadratów. Na każdym z nich napisana będzie nazwa danego wykonawcy. Klikanie na jeden z nich przejdzie do widoku albumów danego wykonawcy.

\begin{figure}[H]
	\centering
	\includegraphics[width=1\linewidth]{images/mockup2_albumy}
	\caption{\centering{Mockup widoku albumów danego wykonawcy}}
	\label{fig:mockup2albumy}
\end{figure}

Widok albumów jest przedstawiony na rysunku nr.~\ref{fig:mockup2albumy}. Widok będzie podobny do widoku wykonawców. Różni się on tym, że na "kafelkach", będą pokazane zdjęcia poszczególnych albumów. Pod "kafelkami", znajdują się nazwy danych albumów.

\begin{figure}[H]
	\centering
	\includegraphics[width=1\linewidth]{images/mockup2_utwory}
	\caption{\centering{Mockup widoku wyboru utworu}}
	\label{fig:mockup2utwory}
\end{figure}


Rysunek nr.~\ref{fig:mockup2utwory} przedstawia ekran pokazujący się po wybraniu albumu. Po wejściu na jakiś album zaprezentowane zostaną zawarte w nim utwory. W lewym górnym kwadrat to zdjęcie danego albumu, a obok niego jest kilka informacji o albumie jak wykonawca, data, tytuł, w postaci tekstu. Dłuższe paski zawarte na dole to lista piosenek, w postaci przycisków z napisanymi, tytułami które można kliknąć, aby daną piosenkę włączyć.

\begin{figure}[H]
	\centering
	\includegraphics[width=1\linewidth]{images/mockup3_odtwarzacz}
	\caption{\centering{Mockup widoku wyboru utworu}}
	\label{fig:mockup3odtwarzacz}
\end{figure}

Wygląd interfejsu odtwarzacza został zaprezentowany na rysunku nr.\ref{fig:mockup2odtwarzacz}. W widoku horyzontalnym po lewej stronie na górze znajduje się obraz albumu a na dole pod obrazem będzie nazwa utworu, po prawej stronie na górze znajduje się pasek przewijania w formie soundwave któy jest wyciągany z pliku muzycznego a na dole pod soundwave znajdują się przyciski pozwalające na manipulację utworem jak np. na zatrzymanie go lub przewinięcie. W widoku horyzontalnym na samym dole znajduje się zdjęcie albumu, poniżej nazwa utworu, potem soundwave a na końcu przyciski manipulacyjne.
