
%╔════════════════════════════╗
%║		Szablon wykonał		  ║
%║	mgr inż. Dawid Kotlarski  ║
%║		  06.10.2024		  ║
%╚════════════════════════════╝

\documentclass[12pt,a4paper]{mwart}
\usepackage[utf8]{inputenc}
\usepackage{polski}
\usepackage[T1]{fontenc}
\usepackage{amsmath}
\usepackage{amsfonts}
\usepackage{amssymb}
\usepackage{graphicx}
\usepackage{array}
\usepackage{multirow}
\usepackage{geometry}
\usepackage{tabularray}

\geometry{margin=2cm}

\renewcommand{\arraystretch}{1.2}

\begin{document}
	
\begin{center}
	\Huge RAPORT
\end{center}

\begin{table}[h!]
	\centering
	\begin{tblr}{
			width = \linewidth,
			colspec = {Q[156]Q[156]Q[156]Q[156]Q[156]Q[156]},
			row{1} = {c},
			column{4} = {c},
			column{6} = {c},
			cell{1}{1} = {c=6}{0.936\linewidth},
			cell{2}{2} = {c=5}{0.803\linewidth},
			cell{3}{2} = {c=5}{0.803\linewidth},
			cell{4}{2} = {c},
			cell{5}{2} = {c=5}{0.803\linewidth},
			hline{1,6} = {1}{-}{leftpos = 1, rightpos = 1},
			hline{1,6} = {2}{-}{leftpos = 1, rightpos = 1},
			hline{2,2} = {1}{-}{leftpos = 1, rightpos = 1},
			hline{2,2} = {2}{-}{leftpos = 1, rightpos = 1},
			vline{1,1} = {1}{-}{abovepos = 1, belowpos = 1},
			vline{1,1} = {2}{-}{abovepos = 1, belowpos = 1},
			vline{7,1} = {1}{-}{abovepos = 1, belowpos = 1},
			vline{7,1} = {2}{-}{abovepos = 1, belowpos = 1},
			hlines,
			vlines,
		}
		{AKADEMIA NAUK STOSOWANYCH W NOWYM SĄCZU\\Wydział Nauk Inżynieryjnych, Katedra informatyki} &  &  &  &  &  \\
		Przedmiot:  & Programowanie urządzeń mobilnych – projekt, mgr inż. Dawid Kotlarski          &  &  &  &  \\
		Temat:      & Raptor                                                         &  &  &  &  \\
		Grupa:      & IS-2(s)P3  & Nr raportu: & 10 & Data: & 14.01.2025 \\
		Osoby:      & Stanek Mateusz, Szołdra Dawid, Filip Wąchała                                              &  &  &  &            
	\end{tblr}
\end{table}


\section{Wykonane zadania}
\begin{itemize}
	\item Redesign layoutów - layouty przy telefonie obróconym poziomo.
	
	\item Dodanie rozdziału 6
\end{itemize}

\section{Niewykonane zadania}

\begin{itemize}
	\item Brak
\end{itemize}

\section{Napotkane problemy}

\begin{itemize}
	\item Brak
\end{itemize}

\section{Zadania na kolejny tydzień}

\begin{itemize}
	
	\item Dalsza praca nad dokumentacją
\end{itemize}

\section{Stan dokumentacji projektowej}
\begin{itemize}
	\item W rozdziale 6 opisana została instrukcja użytkowania aplikacji.
\end{itemize}

\end{document}
